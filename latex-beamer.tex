\documentclass{beamer}
\usepackage{graphicx}
\usepackage{listings}
\usepackage{xcolor}
\usetheme{Pittsburgh}

\definecolor{maincs}{RGB}{255,0,0}
\definecolor{secondarycs}{RGB}{255,179,246}

\lstset{
    language=[LaTeX]TeX,
    xleftmargin=2cm,
    escapeinside={*@}{@*},
    basicstyle=\ttfamily\small,
    columns=fullflexible,
    breaklines=true,
    texcsstyle=*\color{maincs},
    texcs={documentclass,begin,end,chapter,section,subsection,label,alpha,part},
    moretexcs=[2]{usepackage,input},
    texcsstyle=*[2]{\color{secondarycs!80!black}},
}

\beamertemplatenavigationsymbolsempty
\title{A presentation in \LaTeX\ Beamer on \LaTeX\ Beamer}
\author{Caleb Jhones}
\date{20 October 2016}

\begin{document}
\begin{frame}
    \maketitle
\end{frame}

\begin{frame}[fragile]
\frametitle{Tabular}
\begin{itemize}
    \item Use the \texttt{tabular} environment
        \begin{itemize}
        \item \verb+&+ acts as an alignment character, and \verb+\\+ acts as a newline
        \end{itemize}
    \pause
    \item In order to get a table like this: \\
        \begin{table}
        \centering
        \begin{tabular}{ |c|c| }
        \hline
        stuff & stuff \\ \hline
        stuff & stuff \\
        \hline
        \end{tabular}
        \end{table}
    \pause
    \item Use code that looks like this:
    \begin{lstlisting}
    \begin{tabular}{ |c|c| }
    \hline
    stuff & stuff \\ \hline
    stuff & stuff \\
    \hline
    \end{tabular}
    \end{lstlisting}
\end{itemize}
\end{frame}

\begin{frame}[fragile]
\frametitle{Graphicx}
\begin{itemize}
    \item To enable including pictures, include \texttt{graphicx} in your preamble:
    \begin{lstlisting}
    \usepackage{graphicx}
    \end{lstlisting}
    Then in the body of your document:
    \begin{lstlisting}
    \includegraphics[width=4cm]{pic.png}
    \end{lstlisting}
    where the argument(s) and file extension are optional
\end{itemize}
\end{frame}

\begin{frame}
\frametitle{What is \texttt{beamer}?}
\begin{itemize}
    \item So you've taken what Nick said to heart. But how do you get pretty slides like those that he used???
    \pause
    \item The answer is to use \texttt{beamer}! It uses all of the same sorts of syntax (for math, tables, pictures, etc.), and includes control
sequences for making bullet points, slides, et al.
\end{itemize}
\end{frame}

\begin{frame}[fragile]
\frametitle{The basics}
\begin{itemize}
    \item When beginning your \texttt{.tex} document, use the \texttt{beamer} document class to have access to the slideshow-specific control sequences
    \pause
    \item You will also want to chose a theme by typing \verb+\usetheme{theme goes here}+ after the \verb+\usepackage+ section
    \begin{itemize}
        \item I personally prefer the \texttt{Pittsburgh} theme, but others around LUG like \texttt{Luebeck}, \texttt{Rochester}, \texttt{Metropolis}, and others. Ask around or run a quick search to find one you like
    \end{itemize}
    \pause
    \item Each slide is within the \texttt{frame} environment, and you give slides titles by calling \verb+\frametitle{Title goes here}+
\end{itemize}
\end{frame}

\begin{frame}[fragile]
\frametitle{Title slide}
\begin{itemize}
    \item A good slideshow needs a titlepage, so we use the \verb+\maketitle+ command for this. It is placed in its own \texttt{frame} like so:
    \begin{lstlisting}
    \begin{frame}
        \maketitle
    \end{frame}
    \end{lstlisting}
    \pause
    \item This title slide gets its info from the \texttt{title}, \texttt{author}, \texttt{date}, etc fields:
    \begin{lstlisting}
    \title{Clever title}
    \author{John Smith}
    \date{7 February 1941}
    \end{lstlisting} %leslie lanport's birthday
\end{itemize}
\end{frame}

\begin{frame}[fragile]
\frametitle{Bullet points}
\begin{itemize}
    \item Use the \texttt{itemize} control sequence just like in any other \LaTeX\ document
    \pause
    \item To get pauses between different bullet points within a slide, use the \verb+\pause+ control sequence between \verb+\item+s
\end{itemize}
\end{frame}

\begin{frame}[fragile]
\frametitle{Graphics and tables}
\begin{itemize}
    \item Graphics are easy! Just use the \texttt{graphicx} package already mentioned
    \pause
    \item Tables are equally easy to use--also refer to the section above
    \pause
    \item For graphics and tables, use \verb+\caption{Cool caption}+ to add captions
\end{itemize}
\end{frame}

\begin{frame}[fragile]
\frametitle{"Full" example}
\begin{lstlisting}
\documentclass{beamer}
\usetheme{Pittsburgh}

\begin{document}
    % Title slide would go here
    \begin{frame}
    \frametitle{My first slide!}
    \begin{itemize}
        \item A bullet point
        \pause
        \item Another one!
    \end{itemize}
\end{document}
\end{lstlisting}
\end{frame}

\begin{frame}
\frametitle{Take-aways and questions}
\begin{itemize}
    \item \texttt{beamer} is very easy to use: it adds small amounts of functionality to what you will become used to in \LaTeX\ to allow you to make slides
    \pause
    \item If you need an example, look to the source of most of the presentations on \texttt{lug.mines.edu}
    \pause
    \begin{figure}
    \centering
    \includegraphics[width=6cm]{tex-lion}
    \end{figure}
\end{itemize}
\end{frame}

\end{document}
